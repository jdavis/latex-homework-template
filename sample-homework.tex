%%%%%%%%%%%%%%%%%%%%%%%
% PHY 280 Homework Style Sheet
% Adapted from Josh Davis's great work: https://github.com/jdavis/latex-homework-template/blob/master/homework.tex
%
% Make sure that mathhomework.sty is in your working directory!
%
%%%%%%%%%%%%%%%%%%%%%%%


\documentclass{article}
\usepackage{mathhomework}


%%%%%% Edit these variables %%%%%%

\hmwkAuthor{Jane L.\ Jankowski} 		% Put your name here

\hmwkNumber{1} 					% What Homework number is this? 

\hmwkDueDate{September 4, 2016} 	% When is this due?

\hmwkClass{Math Methods}			% Put the name of the Class here!
 
%%%%%%%%%%%%%%%%%%%%%%

\begin{document}
\maketitle
\thispagestyle{empty}
\pagebreak
\setcounter{page}{1}

%%%%%%%%% Homework Problems Go  Below Here %%%%%%%%%%%%%%
%
% Enclose homework problems in the homeworkProblem environment
% 
% Use \solution to mark the problem solution
%
% Use \part to start a seperate part of the problem
%
% Use \pagebreak to start a new page.  
%
%%%%%%%%%%%%%%%%%%%%%%%%%%%%%%%%%%%%%%%%%%
%
% The package already has the amsmath extensions built in. To create a 
% multi-line equation, use the split environment: 
%
% \begin{equation*}		<----- The * removes automatic equation numbering
%	\begin{split}		<----- Allows for equations to be alinged at the &= sign
%	
%	y &= x^2+2		<----- Note the &=
%	   &= 6			<----- On every line that needs to be aligned. 
%
%	\end{split}
% \end{equation*}		<----- If you begin an environment, end it!
%
%%%%%%%%%%%%%%%%%%%%%%%%%%%%%%%%%%%%%%%%%%


\begin{homeworkProblem}
    Give an appropriate positive constant $c$ such that $f(n) \leq c \cdot
    g(n)$ for all $n > 1$.

    \begin{enumerate}
        \item $f(n) = n^2 + n + 1$, $g(n) = 2n^3$
        \item $f(n) = n\sqrt{n} + n^2$, $g(n) = n^2$
        \item $f(n) = n^2 - n + 1$, $g(n) = n^2 / 2$
    \end{enumerate}

    \solution

    We solve each solution algebraically to determine a possible constant
    $c$.
    \\

   \part

    \begin{equation*}
        \begin{split}
            n^2 + n + 1 &=
            \\
            &\leq n^2 + n^2 + n^2
            \\
            &= 3n^2
            \\
            &\leq c \cdot 2n^3
        \end{split}
    \end{equation*}

    Thus a valid $c$ could be when $c = 2$.
    \\

\part

    \begin{equation*}
        \begin{split}
            n^2 + n\sqrt{n} &=
            \\
            &= n^2 + n^{3/2}
            \\
            &\leq n^2 + n^{4/2}
            \\
            &= n^2 + n^2
            \\
            &= 2n^2
            \\
            &\leq c \cdot n^2
        \end{split}
    \end{equation*}

    Thus a valid $c$ is $c = 2$.
    \\
\pagebreak

  \part

    \begin{equation*}
        \begin{split}
            n^2 - n + 1 &=
            \\
            &\leq n^2
            \\
            &\leq c \cdot n^2/2
        \end{split}
    \end{equation*}

    Thus a valid $c$ is $c = 2$.

\end{homeworkProblem}

\pagebreak

\begin{homeworkProblem}
    Prove a polynomial of degree $k$, $a_kn^k + a_{k - 1}n^{k - 1} + \hdots
    + a_1n^1 + a_0n^0$ is a member of $\Theta(n^k)$ where $a_k \hdots a_0$
    are nonnegative constants.

    \begin{proof}
        To prove that $a_kn^k + a_{k - 1}n^{k - 1} + \hdots + a_1n^1 +
        a_0n^0$, we must show the following:

        \begin{equation*}
            \exists c_1 \exists c_2 \forall n \geq n_0,\ {c_1 \cdot g(n) \leq
            f(n) \leq c_2 \cdot g(n)}
        \end{equation*}

        For the first inequality, it is easy to see that it holds because no
        matter what the constants are, $n^k \leq a_kn^k + a_{k - 1}n^{k - 1} +
        \hdots + a_1n^1 + a_0n^0$ even if $c_1 = 1$ and $n_0 = 1$.  This
        is because $n^k \leq c_1 \cdot a_kn^k$ for any nonnegative constant,
        $c_1$ and $a_k$.
        \\

        Taking the second inequality, we prove it in the following way.
        By summation, $\sum\limits_{i=0}^k a_i$ will give us a new constant,
        $A$. By taking this value of $A$, we can then do the following:

        \begin{equation*}
            \begin{split}
                a_kn^k + a_{k - 1}n^{k - 1} + \hdots + a_1n^1 + a_0n^0 &=
                \\
                &\leq (a_k + a_{k - 1} \hdots a_1 + a_0) \cdot n^k
                \\
                &= A \cdot n^k
                \\
                &\leq c_2 \cdot n^k
            \end{split}
        \end{equation*}

        where $n_0 = 1$ and $c_2 = A$. $c_2$ is just a constant. Thus the
        proof is complete.
    \end{proof}
\end{homeworkProblem}

\pagebreak

%%%%%%%%%%%%%%%%%%%%%%%%%%%%%%%%%%%%%
%
% Non sequential homework problems
%
% Use \displaystyle for larger symbols. 
% However, this may break up lines in an ugly way. Use your discretion. 
% 
%%%%%%%%%%%%%%%%%%%%%%%%%%%%%%%%%%%%%

% Jump to problem 18
\begin{homeworkProblem}[18]
    Evaluate $\displaystyle\sum_{k=1}^{5} k^2$ and $\sum_{k=1}^{5} (k - 1)^2$.

    Find $\displaystyle\deriv{}{x} (x^4 + 3x^2 - 2)$ and $\dpderiv[2]{}{\alpha} \log\left|\frac{\alpha^2-7}{x}\right|$. 
\end{homeworkProblem}

% Continue counting to 19
\begin{homeworkProblem}
Here is my code: 
\begin{lstlisting}[language=Python]
```
This calculates the logarithm of a user-supplied number
Jane Jankowski, 2017
```
import numpy as np

UserInput = input('Please enter x')
if number <= 0:
	print('Choose a number greater than zero!') # Who would do this?
else:
	print('Log({}) = {:4f}'.format(UserInput, np.log(UserInput)))	
\end{lstlisting}
\end{homeworkProblem}

% Go back to where we left off
\begin{homeworkProblem}[3]
    Evaluate the integrals
    $\displaystyle\int_0^1 (1 - x^2) \dx$
    and
    $\int_1^{\infty} \frac{1}{x^2} \dx$.
\end{homeworkProblem}

\end{document}